
\documentclass[./main.tex]{subfiles}


\begin{document}



\section{Wprowadzenie}

\begin{frame}{Temat prezentacji}
    Prezentacja przedstawia analizę zanonimizowanych danych z serwisów Stack Exchange. Każdy taki serwis to forum tematyczne, w którym użytkownicy mogą dodawać posty, komentować, głosować na najlepsze odpowiedzi i zdobywać odznaki.
    
\end{frame}

\begin{frame}{Zagadnienia badawcze}
    W pracy zbadano następujące zagadnienia: 
    
    \begin{itemize}
        \item popularność warzenia piwa na tle innych alkoholi na forum \textbf{\textit{Homebrewing}} \pause
        \item wpływ pandemii COVID-19 na podróżowanie na przykładzie forum podróżniczego \textbf{\textit{Travel}} \pause  
        \item porównanie użytkowników Androida i iOS-a na podstawie forów \textbf{\textit{Android}} i \textbf{\textit{Apple}}
    \end{itemize}
\end{frame}


\begin{frame}{Narzędzia badawcze}
    Do opracowania danych wykorzystano język Python wraz z bibliotekami:
    \begin{itemize}
        \item \textit{pandas} - ramki danych
        \item \textit{numpy} - operacje na wektorach i macierzach
        \item \textit{matplotlib} - wykresy
        \item \textit{scipy} - wybrane algorytmy np. interpolacja
    \end{itemize}
\end{frame}

\end{document}